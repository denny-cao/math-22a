\documentclass[11pt]{scrartcl}
\usepackage[sexy]{../../evan}
\usepackage{graphicx}

\definecolor{dg}{RGB}{2,101,15}
\newtheoremstyle{dotlessP}{}{}{}{}{\color{dg}\bfseries}{}{ }{}
\theoremstyle{dotlessP}
\newtheorem{property}[theorem]{Property}

\newtheoremstyle{dotlessN}{}{}{}{}{\color{teal}\bfseries}{}{ }{}
\theoremstyle{dotlessN}
\newtheorem{notation}[theorem]{Notation}
% Shortcuts
\DeclarePairedDelimiter\ceil{\lceil}{\rceil} % ceil function

\DeclarePairedDelimiter\paren{(}{)} % parenthesis

\newcommand{\df}{\displaystyle\frac} % displaystyle fraction
\newcommand{\qeq}{\overset{?}{=}} % questionable equality

\newcommand{\Mod}[1]{\;\mathrm{mod}\; #1} % modulo operator

\newcommand{\comp}{\circ} % composition

% Text Modifiers
\newcommand{\tbf}{\textbf}
\newcommand{\tit}{\textit}

% Sets
\DeclarePairedDelimiter\set{\{}{\}}
\newcommand{\unite}{\cup}
\newcommand{\inter}{\cap}

\newcommand{\reals}{\mathbb{R}} % real numbers: textbook is Z^+ and 0
\newcommand{\ints}{\mathbb{Z}}
\newcommand{\nats}{\mathbb{N}}
\newcommand{\complex}{\mathbb{C}}
\newcommand{\tots}{\mathbb{Q}}

\newcommand{\powset}{\mathcal{P}} % Power Set

\newcommand{\degree}{^\circ}

% Counting
\newcommand\perm[2][^n]{\prescript{#1\mkern-2.5mu}{}P_{#2}}
\newcommand\comb[2][^n]{\prescript{#1\mkern-0.5mu}{}C_{#2}}

% Relations
\newcommand{\rel}{\mathcal{R}} % relation

\setlength\parindent{0pt}

% Directed Graphs
\usetikzlibrary{arrows}
\tikzset{vertex/.style = {shape=circle,draw,minimum size=2em}}
\tikzset{svertex/.style = {shape=circle,draw,minimum size=.05em,font=\tiny}}
\tikzset{edge/.style = {->,> = latex'}}
\tikzset{dedge/.style = {-> = latex'}}
\tikzset{dot/.style={inner sep=1.5pt,circle,draw,fill}}

% Contradiction
\newcommand\contradiction{\mathbin{\mathpalette\xhash\relax}}
\newcommand{\xhash}[2]{\ooalign{%
Ê $#1\xxhash{#1}{-45}$\cr
Ê $#1\xxhash{#1}{45}$\cr
Ê }%
}
\newcommand{\xxhash}[2]{\rotatebox[origin=c]{#2}{$#1\parallel$}}


\title{MATH 22A: Vector Calculus and Linear Algebra}
\subtitle{PSet 0}
\author{Denny Cao}
\date{\today}
%++++++++++++++++++++++++++++++++++++++++
% title stuff
\usepackage{titling}
\renewcommand\maketitlehooka{\null\mbox{}\vfill}
\renewcommand\maketitlehookd{\vfill\null}
\makeatletter
\renewcommand{\maketitle}{\bgroup\setlength{\parindent}{0pt}
	\begin{flushleft}
		\large\textbf{\@title} \\ \vskip 0.2cm
		\begingroup
		\fontsize{14pt}{12pt}\selectfont
		\title
		\\
		\normalsize Lecture 3: Row Reduction and Row Echelon Forms --- September 11, 2023
		\endgroup \vskip 0.3cm
		\normalsize Pset Due: September 13, 2023 \hfill\rlap{}\textbf{Denny Cao} \\ \vskip 0.1cm
		\hrulefill
	\end{flushleft}\egroup
}
\makeatother

\renewcommand{\theques}{\thesection.\alph{ques}} % Change subtheo counter for alpha output
\declaretheorem[style=basehead,name=Answer,sibling=theorem]{ans}
\renewcommand{\theans}{\thesection.\alph{ans}}
\begin{document}
\maketitle
\pagestyle{plain}
\section{Announcements}
\begin{itemize}
	\item Office Hours: Monday, 4-6 and 7-9 in Sever 306
	\item Check website for Office Hours on Tuesday
	\item Office Hours w/Prof: Tuesday, 3-4 in Science Center 504 and Wednesday, 4-5 in Science Center 504
\end{itemize}
\section{Row Echelon Form}
\begin{definition}
	[Row Echelon Form]
	A matrix is in \textit{row echelon form} if:
	\begin{enumerate}
		\item All rows consisting of only zeros are at the bottom.
		\item The leading coefficient (pivot) of a non-zero row is always strictly to the right of the leading coefficient of the row above it.
	\end{enumerate}
\end{definition}
\begin{remark}
	Row echelon form is when the matrix is an upper triangular matrix.
\end{remark}
\begin{definition}
	[Reduced Row Echelon Form]	
	A matrix in row echelon form is in \textit{reduced row echelon form} if:
	\begin{enumerate}
		\item The leading entries are all 1.
		\item Each leading 1 is the only non-zero entry in its column.
	\end{enumerate}
\end{definition}
\begin{remark}
	Reduced row echelon is when the main diagonal is full with 1s and every other entry is 0.
\end{remark}
\begin{fact}
	Each matrix is row equivalent to one and only one reduced echelon matrix.
\end{fact}
\begin{definition}
	[Pivot Position/Column]
	A \textit{pivot position} in a matrix $A$ is a location that corresponds to a leading 1 in the reduced row echelon form. A \textit{pivot column} is a column that has a pivot position.
\end{definition}
\begin{algorithm}
	[Gaussian Elimination]
	\
	\begin{enumerate}
		\item The leftmost non-zero column is a pivot column.
		\item Select a non-zero entry in the pivot column as pivot. Interchange rows to move this entry to the top.
		\item Use elementary row operations to create zeros below the pivot.
		\item Apply (1)--(3) to the submatrix obtained by removing pivot row and column. Repeat until all non-zero rows are modified.
		\item Beginning with rightmost pivot and working upwards and to the left, create 0's above each pivot. Scale pivot position entries to be 1.
	\end{enumerate}
\end{algorithm}
\begin{theorem}
	Given a matrix $A$, Gaussian Elimination produces $\text{rref}(A)$ and there is a unique  $\text{rref}(A)$.
\end{theorem}
\begin{proof}
	To get to $\text{rref}(A)$, we used elementary row operations. We can invert these row operations from  $\text{rref}(A)$ to the original matrix.
\end{proof}
\begin{definition}
	[Basic, Free]
	The variables corresponding to the pivot columns in an augmented matrix are called \textit{basic} and the others are called \textit{free}.
\end{definition}
\begin{fact}
	A linear system is \textbf{consistent} if and only if the \textbf{rightmost column in the augmented matrix is not a pivot column.} If the system is \textbf{consistent}, then it has a textbf{unique solution} when there are \textbf{no free variables} and \textbf{infinite solutions} when there is one or more free variables.
\end{fact}
\begin{itemize}
	\item System is inconsistent if any non-zero coefficients in the rightmost column in the augmented matrix is not zero in rows with all zeros.
\end{itemize}
\begin{definition}
	[Column/Row Vector]
	A matrix with 1 column is called a \textit{column} vector. A matrix with one row is called a \textit{row vector}.
\end{definition}
\begin{itemize}
	\item We can add vectors of the same size by adding components:
	\[
		\vec{u} + \vec{v} =
		\begin{pmatrix}
			u_1 \\
			\vdots \\
			u_n
		\end{pmatrix} + 
		\begin{pmatrix}
			v_1 \\
			\vdots \\
			v_n
		\end{pmatrix} =
		\begin{pmatrix}
			u_1 + v_1 \\
			\vdots \\
			u_n + v_n 
		\end{pmatrix}
	\] 
\item We can scale a vector by a real number $c$ by multiplying components by $c$ :
	\[
		c\vec{u} = c 
		\begin{pmatrix}
			u_1 \\
			\vdots \\
			u_n 
		\end{pmatrix} = 
		\begin{pmatrix}
			cu_1 \\
			\vdots \\
			cu_n 
		\end{pmatrix}
	\] 
\end{itemize}
\begin{definition}
	[Linear Combination]
	Given $\vec{v_1}, \dots, \vec{v_p} \in \reals^n$ and scalars $c_1, \dots, c_p$, the vector:
	\[
		\vec{y} = c_1 \vec{v_1} + \dots + c_p \vec{v_p}
	\] 
	is called the \textit{linear combination} of $\vec{v_1}, \dots, \vec{v_p}$ with weights $c_1, \dots, c_p$.
\end{definition}
\begin{remark}
	A vector equation 
	\[
		\vec{b} = x_1 \vec{a_1} + x_2 \vec{a_2} + \dots + x_n \vec{a_n}
	\] 
	has the same solutions as the augment matrix
	\[
		\begin{bmatrix}{cccc|c}
			\vec{a_1} & \vec{a_2} & \dots & \vec{a_n} & \vec{b}
	\end{bmatrix}
	\] 
\end{remark}
\begin{definition}
	[Span]
	If $\vec{v_1}, \dots, \vec{v_p} \in \reals^n$, then the set of all linear combinations of $\vec{v_1}, \dots, \vec{v_p}$ is called the \textit{span} of $\vec{v_1}, \dots, \vec{v_p}$. We write this set as $\text{Span}\{\vec{v_1}, \dots, \vec{v_p}\}$.
\end{definition}
\begin{itemize}
	\item Span of $\vec{v}$ is the line through $\vec{v}$, as it contains all possible scales of the vector.
	\item Span of two vectors is any point in plane if they are not parallel, as any point can be produced by a linear combination if they are not parallel.
\end{itemize}
\end{document}
