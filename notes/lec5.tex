\documentclass[11pt]{scrartcl}
\usepackage[sexy]{../../evan}
\usepackage{graphicx}

\definecolor{dg}{RGB}{2,101,15}
\newtheoremstyle{dotlessP}{}{}{}{}{\color{dg}\bfseries}{}{ }{}
\theoremstyle{dotlessP}
\newtheorem{property}[theorem]{Property}

\newtheoremstyle{dotlessN}{}{}{}{}{\color{teal}\bfseries}{}{ }{}
\theoremstyle{dotlessN}
\newtheorem{notation}[theorem]{Notation}
% Shortcuts
\DeclarePairedDelimiter\ceil{\lceil}{\rceil} % ceil function

\DeclarePairedDelimiter\paren{(}{)} % parenthesis

\newcommand{\df}{\displaystyle\frac} % displaystyle fraction
\newcommand{\qeq}{\overset{?}{=}} % questionable equality

\newcommand{\Mod}[1]{\;\mathrm{mod}\; #1} % modulo operator

\newcommand{\comp}{\circ} % composition

% Text Modifiers
\newcommand{\tbf}{\textbf}
\newcommand{\tit}{\textit}

% Sets
\DeclarePairedDelimiter\set{\{}{\}}
\newcommand{\unite}{\cup}
\newcommand{\inter}{\cap}

\newcommand{\reals}{\mathbb{R}} % real numbers: textbook is Z^+ and 0
\newcommand{\ints}{\mathbb{Z}}
\newcommand{\nats}{\mathbb{N}}
\newcommand{\complex}{\mathbb{C}}
\newcommand{\tots}{\mathbb{Q}}

\newcommand{\powset}{\mathcal{P}} % Power Set

\newcommand{\degree}{^\circ}

% Counting
\newcommand\perm[2][^n]{\prescript{#1\mkern-2.5mu}{}P_{#2}}
\newcommand\comb[2][^n]{\prescript{#1\mkern-0.5mu}{}C_{#2}}

% Relations
\newcommand{\rel}{\mathcal{R}} % relation

\setlength\parindent{0pt}

% Directed Graphs
\usetikzlibrary{arrows}
\tikzset{vertex/.style = {shape=circle,draw,minimum size=2em}}
\tikzset{svertex/.style = {shape=circle,draw,minimum size=.05em,font=\tiny}}
\tikzset{edge/.style = {->,> = latex'}}
\tikzset{dedge/.style = {-> = latex'}}
\tikzset{dot/.style={inner sep=1.5pt,circle,draw,fill}}

% Contradiction
\newcommand{\contradiction}{{\hbox{%
    \setbox0=\hbox{$\mkern-3mu\times\mkern-3mu$}%
    \setbox1=\hbox to0pt{\hss$\times$\hss}%
    \copy0\raisebox{0.5\wd0}{\copy1}\raisebox{-0.5\wd0}{\box1}\box0
}}}


\title{MATH 22A: Vector Calculus and Linear Algebra}
\subtitle{PSet 1}
\author{Denny Cao}
\date{\today}
%++++++++++++++++++++++++++++++++++++++++
% title stuff
\usepackage{titling}
\renewcommand\maketitlehooka{\null\mbox{}\vfill}
\renewcommand\maketitlehookd{\vfill\null}
\makeatletter
\renewcommand{\maketitle}{\bgroup\setlength{\parindent}{0pt}
	\begin{flushleft}
		\large\textbf{\@title} \\ \vskip 0.2cm
		\begingroup
		\fontsize{14pt}{12pt}\selectfont
		\title
		\\
		\normalsize Lecture 5: Proof Methods --- \today
		\endgroup \vskip 0.3cm
		\normalsize Pset Due: September 20, 2023 \hfill\rlap{}\textbf{Denny Cao} \\ \vskip 0.1cm
		\hrulefill
	\end{flushleft}\egroup
}
\makeatother

\renewcommand{\theques}{\thesection.\alph{ques}} % Change subtheo counter for alpha output
\declaretheorem[style=basehead,name=Answer,sibling=theorem]{ans}
\renewcommand{\theans}{\thesection.\alph{ans}}
\begin{document}
\maketitle
\pagestyle{plain}
\section{Proof Methods}
\begin{definition}
	[Theorem]
	Statement true and verified if the statement $P \to Q$ is true, where $P$ is the assumption and $Q$ is the conclusion.
\end{definition}
\begin{notation}
	[Divides]
	For two integers $a,b$  $a$ is said to divide $b$ if  $\exists c \mid ac = b$. We write this as  $a \mid b$.
\end{notation}
\begin{definition}
	[Direct Proof]
	Proof by case by case consideration or by explicit demonstration why, with the assumption $P$, $Q$ holds.
\end{definition}	
\begin{definition}
	[Binomial Coefficient/Choose]
	$\displaystyle \binom{n}{k} = \frac{n!}{(n-k)!k!}, 0 \leq k \leq n$.
\end{definition}
\begin{example}
	If $n \geq 2, n \in \ints$, then $n^2 = \displaystyle 2 \binom{n}{2} + \binom{n}{1}$.
\end{example}
\begin{proof}
	\begin{align*}
		2 \binom{n}{2} + \binom{n}{1} &= 2\paren*{\frac{n!}{(n-2)!2!}} + \frac{n!}{(n-1)!1!} \\
									  &= \frac{n!}{(n-2)!} + \frac{n!}{n-1}! \\
									  &= n(n-1) + n \\
									  &= n^2 - n + n \\
									  &= n^2
	\end{align*}
	Thus, we have shown that the statement is true directly.
\end{proof}
\begin{example}
	If $x^2 \in \ints \land x^2 \mid x \to x \in \set*{-1,0,1}$.
\end{example}
\begin{proof}
	$x^2 \mid x \to \exists c \in \ints \mid cx^2 = x$.  We consider two cases:

	\textit{Case 1}: $x = 0$. If $x=0$, then the statement that $x^2 \mid x$ is true.

	\textit{Case 2}:  $x \neq 0$. Then, $c = \displaystyle \frac{x}{x^2} = \frac{1}{x} = c$. For $c$ to be an integer, $x = -1 \lor x = 1$.
	\\

	Thus, we have shown by direct proof that the statement is true.
\end{proof}
\begin{definition}
	[Contrapositive Proof]
	We are trying to prove $P \to Q$. We can also show that $\neg Q \to \neg P$, as they have the same truth value.
\end{definition}
\begin{definition}
	[Congruence Modulo $n$]
	$a \equiv b \mod (n) \to (a - b) \mid n$ 
\end{definition}
\begin{theorem}
	$a \equiv b \mod (n) \to a \mod (n) = b \mod (n)$
\end{theorem}
\begin{example}
\end{example}
\begin{definition}
	[Proof by Contradiction]
	We are trying to prove that $P \to Q$. By contradiction, we assume $P \to \neg Q$ and reach a contradiction. Thus, the opposite must be true: $P \to Q$. 
\end{definition}
\begin{example}
	Prove that $\sqrt[3] 2$ is not a rational number.
\end{example}
\begin{proof}
	Assume for purposes of contradiction that $\sqrt[3] 2$ is a rational number, or that $\sqrt[3] 2 = p/q, p,q \in \ints$ such that $p/q$ is in reduced form.
	$2 = \displaystyle\frac{p^3}{q^3} \to p^3 = 2q^3 \to$ $p$ is an even number $\to p = 2k, k \in \ints \to p^3 = 8k^3 \to 8k^3 = 2q^3 \to q^3 = 4k^3 \to$  $q$ is even $\to q = 4l$. Thus,  $p/q = \frac{2k}{4l} = \frac{k}{2l}$. $\contradiction$.

	We reach a contradiction, as $p/q$ is not in reduced form. Thus, we have shown by contradiction that $\sqrt[3] 2$ is not a rational number.
\end{proof}
\section{Set Theory}
\end{document}
