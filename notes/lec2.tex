\documentclass[11pt]{scrartcl}
\usepackage[sexy]{../../evan}
\usepackage{graphicx}

\definecolor{dg}{RGB}{2,101,15}
\newtheoremstyle{dotlessP}{}{}{}{}{\color{dg}\bfseries}{}{ }{}
\theoremstyle{dotlessP}
\newtheorem{property}[theorem]{Property}

\newtheoremstyle{dotlessN}{}{}{}{}{\color{teal}\bfseries}{}{ }{}
\theoremstyle{dotlessN}
\newtheorem{notation}[theorem]{Notation}
% Shortcuts
\DeclarePairedDelimiter\ceil{\lceil}{\rceil} % ceil function

\DeclarePairedDelimiter\paren{(}{)} % parenthesis

\newcommand{\df}{\displaystyle\frac} % displaystyle fraction
\newcommand{\qeq}{\overset{?}{=}} % questionable equality

\newcommand{\Mod}[1]{\;\mathrm{mod}\; #1} % modulo operator

\newcommand{\comp}{\circ} % composition

% Text Modifiers
\newcommand{\tbf}{\textbf}
\newcommand{\tit}{\textit}

% Sets
\DeclarePairedDelimiter\set{\{}{\}}
\newcommand{\unite}{\cup}
\newcommand{\inter}{\cap}

\newcommand{\reals}{\mathbb{R}} % real numbers: textbook is Z^+ and 0
\newcommand{\ints}{\mathbb{Z}}
\newcommand{\nats}{\mathbb{N}}
\newcommand{\complex}{\mathbb{C}}
\newcommand{\tots}{\mathbb{Q}}

\newcommand{\powset}{\mathcal{P}} % Power Set

\newcommand{\degree}{^\circ}

% Counting
\newcommand\perm[2][^n]{\prescript{#1\mkern-2.5mu}{}P_{#2}}
\newcommand\comb[2][^n]{\prescript{#1\mkern-0.5mu}{}C_{#2}}

% Relations
\newcommand{\rel}{\mathcal{R}} % relation

\setlength\parindent{0pt}

% Directed Graphs
\usetikzlibrary{arrows}
\tikzset{vertex/.style = {shape=circle,draw,minimum size=2em}}
\tikzset{svertex/.style = {shape=circle,draw,minimum size=.05em,font=\tiny}}
\tikzset{edge/.style = {->,> = latex'}}
\tikzset{dedge/.style = {-> = latex'}}
\tikzset{dot/.style={inner sep=1.5pt,circle,draw,fill}}

% Contradiction
\newcommand\contradiction{\mathbin{\mathpalette\xhash\relax}}
\newcommand{\xhash}[2]{\ooalign{%
Ê $#1\xxhash{#1}{-45}$\cr
Ê $#1\xxhash{#1}{45}$\cr
Ê }%
}
\newcommand{\xxhash}[2]{\rotatebox[origin=c]{#2}{$#1\parallel$}}

\title{MATH 22A: Vector Calculus and Linear Algebra}
\subtitle{PSet 0}
\author{Denny Cao}
\date{\today}
%++++++++++++++++++++++++++++++++++++++++
% title stuff
\usepackage{titling}
\renewcommand\maketitlehooka{\null\mbox{}\vfill}
\renewcommand\maketitlehookd{\vfill\null}
\makeatletter
\renewcommand{\maketitle}{\bgroup\setlength{\parindent}{0pt}
	\begin{flushleft}
		\large\textbf{\@title} \\ \vskip 0.2cm
		\begingroup
		\fontsize{14pt}{12pt}\selectfont
		\title
		\\
		\normalsize Lecture 2: Proofs --- September 8, 2023 
		\endgroup \vskip 0.3cm
		\normalsize Pset Due: September 13, 2023 \hfill\rlap{}\textbf{Denny Cao} \\ \vskip 0.1cm
		\hrulefill
	\end{flushleft}\egroup
}
\makeatother

\renewcommand{\theques}{\thesection.\alph{ques}} % Change subtheo counter for alpha output
\declaretheorem[style=basehead,name=Answer,sibling=theorem]{ans}
\renewcommand{\theans}{\thesection.\alph{ans}}
\begin{document}
\maketitle
\pagestyle{plain}
\section{Announcements}
\begin{itemize}
	\item Office Hours rooms change; check website. Currently, Sever 306 check back later.
\end{itemize}
\section{Review}
\subsection{Matrices}
\begin{definition}
	[Matrix]	
	A \textit{matrix} is $m \times n$, where $m$ is the number of rows, $n$ is the number of columns. The numbers in a matrix are called \textit{coefficients}.
\end{definition}
\begin{property}
	[Row Operations]
	\
	\begin{itemize}
		\item You can scale a row by a real number
		\item Swap two rows
		\item Add multiple of a row to another
	\end{itemize}
\end{property}
\section{Introductory Set Theory}
\subsection{Notation}
\begin{definition}
	[Set]
	A \textit{set} is a collection of objects.
\end{definition}
\begin{notation}
	[Member]
	$p \in A$ means $p$ is in, or is a member, of the set  $A$.  $p \not\in A$ means $p$ is not in, or is not a member, of the set $A$.
\end{notation}
\begin{notation}
	[Complement]
	$\overline{A}$ means everything that is not in $A$. 
\end{notation}
\begin{definition}
	[Prime Number]
	A positive integer $p$ is \textit{prime} if $p$ is divisible by only itself and 1.
\end{definition}
\begin{example}
	Does there exist 100 consecutive composite numbers?
\end{example}
\begin{proof}
	Let $k \in \set*{0 \leq k \leq 100, k \in \ints}$. Then, $\forall k$ :
	\[
		\frac{101! + k}{k} \in \ints
	\] 
	As $k$ is a factor of $101!$ and $k$ is a factor of  $k$. Thus, there does exist 100 consecutive composite numbers starting from $101!$.
\end{proof}
\begin{notation}
	[Set Builders]
	\
	\begin{itemize}
		\item $\nats = \set*{0,1,2,3,...}$
		\item $\ints = \set*{...,-1,0,1,...}$
		\item $\nats = \set*{p \in \ints \mid p \geq 0}$
	\end{itemize}
\end{notation}	
\begin{notation}
	[Cardinality]
	The cardinality is the size of the set, denoted by $|A|$. When  $A$ is finite, the cardinality is the number of elements.
\end{notation}
\begin{definition}
	[Universal Set/Empty Set]
	The empty set, $\emptyset$, is a set with no elements.
\end{definition}
\begin{definition}
	[Cartesian Product]
	$A \times B = \set*{(a,b) \mid a \in A \land b \in B}$
\end{definition}
\begin{example}
	[Planes]
	$\reals$ is all reals in the 1-dimensional plane. $\reals^2 = \reals \times \reals$, which is the cartesian product and creates the 2-dimensional plane by creating all ordered pairs of $(x,y)$. Same for $\reals^3$.
\end{example}
\begin{notation}
	[Subset]
	$B \subseteq A$ denotes that $B$ is a part of $A$.
\end{notation}
\begin{itemize}
	\item $\emptyset \subseteq A \land A \subseteq A, \forall A$
\end{itemize}
\begin{definition}
	[Power Set]
	The power set of a set $A$ is the set of all subsets of $A$, denoted $\mathcal{P}(A)$.
\end{definition}
\begin{remark}
	[Cardinality of the Power Set]
	$|A| = n \to |\powset(A)| = 2^n$
\end{remark}
\begin{definition}
	[Union]
	The union of two sets  $ A$ and $B$ denoted $A \unite B$ is everything in $A$ or $B$. $A \unite B = \set*{c \mid c \in A \lor A \in B}$.
\end{definition}
\begin{definition}
	[Intersection]
	The intersection of two sets $A$ and $B$ denoted $A \inter B$ is everything in both $A$ and $B$. $A \inter B = \set*{c \mid c \in A \land A \in B}$.
\end{definition}
\subsection{Russell's Paradox}
Paradox assertions without truth values.
\section{Proofs}	
\begin{definition}
	[Existence Statement]
	$\exists x \in A \mid P(x)$.
\end{definition}
\begin{itemize}
	\item To disprove: Prove the negative. $\forall x \in A \mid \neg P(x)$
	\item To prove: Find an $x \mid P(x)$.
\end{itemize}	
\begin{definition}
	[Universal Statement]
	$\forall x \in A \mid P(x)$.
\end{definition}
\begin{itemize}
	\item To disprove: Prove that $\exists x \in A \mid \neg P(x)$.
	\item To prove: Check all  $x$.
\end{itemize}
\begin{definition}
	[Disprove by Contradiction]
	Assume for purposes of contradiction that $P(x)$ is true and then deduce by logic that it is is false by coming to a contradiction, and therefore $P(x)$ is false.
\end{definition}
\begin{example}
	Claim: $\sqrt 2$ is irrational.
\end{example}
\begin{proof}
	By contradiction. Assume for purposes of contradiction that $\sqrt 2$ is rational. Then,  $\sqrt 2$ can be written in the form $p/q$ such that $q$ is the smallest possible number. 
	\\
	\begin{align*}
		\sqrt 2 &= \frac{p}{q} \\
		2 &= \frac{p^2}{q^2} \\
		2q^2 &= p^2
	\end{align*}
	If $p$ is even:
	\begin{align*}
		2q^2 &= 4m^2 \\
		q^2 &= 2m^2 
	\end{align*}
	This means that $q$ is divisible by 2 and not in lowest terms. $\contradiction$

	If  $p$ is odd:
	\begin{align*}
		2q^2 &= 4b^2 + 4b + 1 \\
		q^2 &= b^2 + 2b + 1/2 \\
		q &= \sqrt{b^2 + 2b + 1/2}
	\end{align*}
	This means that  $q$ is not an integer. $\contradiction$
	\end{proof}
\end{document}
