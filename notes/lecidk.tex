\documentclass[11pt]{scrartcl}
\usepackage[sexy]{../../evan}
\usepackage{float}
\usepackage{graphicx}
\usepackage{bm}
\usepackage{pgfplots}
\usetikzlibrary{calc}
\usetikzlibrary{decorations,calligraphy}
\usetikzlibrary{matrix,decorations.pathreplacing, calc, positioning,fit, bending}
\definecolor{dg}{RGB}{2,101,15}
\newtheoremstyle{dotlessP}{}{}{}{}{\color{dg}\bfseries}{.}{ }{}
\theoremstyle{dotlessP}
\newtheorem{property}[theorem]{Property}
\newtheorem{axiom}{Axiom}
\newtheoremstyle{dotlessN}{}{}{}{}{\color{teal}\bfseries}{}{ }{}
\theoremstyle{dotlessN}
\newtheorem{notation}[theorem]{Notation}
% Shortcuts
\DeclarePairedDelimiter\ceil{\lceil}{\rceil} % ceil function

\DeclarePairedDelimiter\paren{(}{)} % parenthesis

\newcommand{\df}{\displaystyle\frac} % displaystyle fraction
\newcommand{\qeq}{\overset{?}{=}} % questionable equality

\newcommand{\Mod}[1]{\;\mathrm{mod}\; #1} % modulo operator

\newcommand{\comp}{\circ} % composition

\newcommand{\lra}{\leftrightarrow}

% Text Modifiers
\newcommand{\tbf}{\textbf}
\newcommand{\tit}{\textit}

% Sets
\DeclarePairedDelimiter\set{\{}{\}}
\newcommand{\unite}{\cup}
\newcommand{\inter}{\cap}

\newcommand{\reals}{\mathbb{R}} % real numbers: textbook is Z^+ and 0
\newcommand{\ints}{\mathbb{Z}}
\newcommand{\nats}{\mathbb{N}}
\newcommand{\complex}{\mathbb{C}}
\newcommand{\tots}{\mathbb{Q}}
\newcommand{\smin}{\setminus}
\newcommand{\degree}{^\circ}
\newcommand{\xor}{\oplus}
\newcommand{\powset}{\mathcal{P}}
% Counting
\newcommand\perm[2][^n]{\prescript{#1\mkern-2.5mu}{}P_{#2}}
\newcommand\comb[2][^n]{\prescript{#1\mkern-0.5mu}{}C_{#2}}

% Relations
\newcommand{\rel}{\mathcal{R}} % relation

\setlength\parindent{0pt}

% Directed Graphs
\usetikzlibrary{arrows}
\tikzset{vertex/.style = {shape=circle,draw,minimum size=2em}}
\tikzset{svertex/.style = {shape=circle,draw,minimum size=.05em,font=\tiny}}
\tikzset{edge/.style = {->,> = latex'}}
\tikzset{dedge/.style = {-> = latex'}}
\tikzset{dot/.style={inner sep=1.5pt,circle,draw,fill}}

% Linear Algebra
\newcommand{\nul}{\text{Nul}}
\newcommand{\col}{\text{Col}}
\newcommand{\row}{\text{Row}}
\newcommand{\adj}{\text{adj}}
\newcommand{\spa}[1]{\text{Span}\set*{#1}}
\newcommand{\mb}[1]{\mathbf{#1}}
\newcommand{\poly}{\mathbb{P}}
\newcommand{\basis}{\mathcal{B}}
\newcommand{\kernel}{\text{ker}}
% Contradiction
\newcommand{\contradiction}{{\hbox{%
    \setbox0=\hbox{$\mkern-3mu\times\mkern-3mu$}%
    \setbox1=\hbox to0pt{\hss$\times$\hss}%
    \copy0\raisebox{0.5\wd0}{\copy1}\raisebox{-0.5\wd0}{\box1}\box0
}}}
\newcommand{\xxhash}[2]{\rotatebox[origin=c]{#2}{$#1\parallel$}}

\hypersetup{
	linkcolor=magenta
}
\title{MATH 22A: Vector Calculus and Linear Algebra}
\subtitle{PSet 1}
\author{Denny Cao}
\date{\today}
%++++++++++++++++++++++++++++++++++++++++
% title stuff
\usepackage{titling}
\renewcommand\maketitlehooka{\null\mbox{}\vfill}
\renewcommand\maketitlehookd{\vfill\null}
\makeatletter
\renewcommand{\maketitle}{\bgroup\setlength{\parindent}{0pt}
	\begin{flushleft}
		\large\textbf{\@title} \\ \vskip 0.2cm
		\begingroup
		\fontsize{14pt}{12pt}\selectfont
		\title
		\\
		\normalsize Eigenvalues and Eigenvectors --- \today
		\endgroup \vskip 0.3cm
		\normalsize Pset Due: November 8, 2023 \hfill\rlap{}\textbf{Denny Cao} \\ \vskip 0.1cm
		\hrulefill
	\end{flushleft}\egroup
}
\renewcommand*\env@matrix[1][*\c@MaxMatrixCols r]{%
  \hskip -\arraycolsep
  \let\@ifnextchar\new@ifnextchar
  \array{#1}}
\makeatother

\renewcommand{\theques}{\thesection.\alph{ques}} % Change subtheo counter for alpha output
\declaretheorem[style=basehead,name=Answer,sibling=theorem]{ans}
\renewcommand{\theans}{\thesection.\alph{ans}}
\begin{document}
\maketitle
\pagestyle{plain}
\section*{Administrivia}
\begin{itemize}
	\item Practice Midterm: 4-5:30 PM at Science Center Lecture Hall B
	\item Review Session 7-9 PM at Jefferson 250 (Today)
\end{itemize}
\section{Review}
\begin{theorem}
	Let $\basis = \set*{b_1, \dots, b_n}$ and $\mathcal{C} = \set*{c_1, \dots, c_n}$ be bases of a vector space $V$. Then $\exists$ a unique $n \times n$ matrix $P \mid [x]_\mathcal{C} = [x]_\basis$.
\end{theorem}
\begin{itemize}
	\item $P$ is the \textbf{change of coordinate matrix from $\basis$ to $\mathcal{C}$}.
		\[
			P = [[b_1]_\mathcal{C}, [b_n]_\mathcal{C}]
		\] 
		Also $P^{-1}$ is the change of coordinates matrix from $\mathcal{C}$ to $\basis$.
	\item Let $V$ be a vector space w/basis $\basis = \set*{b_1, \dots, b_n}$ and $W$ be a vector space with basis $\mathcal{C} = \set*{c_1, \dots, c_n}$. With transformation $T: V \to W$, we want an  $m \times n$ matrix $M$ s.t. $[T(x)]_\mathcal{C} = M[x]_\basis \forall x \in V$. $M$ is the \textbf{matrix of $T$ relative to bases $\basis$ and $\mathcal{C}$}.
		\[
			M = [[T(b_1)]_\mathcal{C}, \dots, [T(b_n)]_\mathcal{C}]
		\] 
\end{itemize}
	\begin{theorem}
		[Diagonal Representation Theorem]
		Suppose $A = PDP^{-1}$ where $D$ is a diagonal matrix and $T(x) = Ax$. If $\basis$ is the basis for $\reals^n$ formed by columns of $P$, then $D$ is the matrix of $T$ relative to $\basis$.
	\end{theorem}
	\begin{itemize}
		\item If $A\land B$ are $n \times n$ matrices, then $A$ is \textbf{similar} to $B$ if $\exists$ an invertible matrix $P$ s.t. $A = PBP^{-1}$.
	\end{itemize}
	
	\begin{definition}
		[Eigenvalue, Eigenvector]
		If $A$ is an $n \times n$ matrix, then $\lambda \in \reals$ is an \textbf{eigenvalue} of $A$ if $\exists$ a non-zero vector $v$, called an \textbf{eigenvector} corresponding to $\lambda$, such that
		\[
		Av = \lambda v
		\] 
	\end{definition}
	\begin{example}
		Let $A = 
\begin{bmatrix}
	1 & 3 \\
	4 & 2
\end{bmatrix}
		$ and $v = 
\begin{bmatrix}
	1 \\
	-1
\end{bmatrix}
		$. Determine if $v$ is an eigenvector of $A$ and find the corresponding eigenvalue.
	\end{example}
	\begin{soln}
		$A\bm{v} = \lambda \bm{v}$. In this case, $\lambda = -2$.
	\end{soln}
	\begin{definition}
		[Eigenspace]
		If $\lambda$ is an eigenvalue of an $n \times n$ matrix $A$, then the set of all solutions $v$ to $Av = \lambda v$ is the \textbf{eigenspace} of $A$ corresponding to $\lambda$.
	\end{definition}
	\begin{itemize}
		\item Eigenspace is $\kernel (A - \lambda I_n)$.
	\end{itemize}
	\begin{example}
		Let $A = 
\begin{bmatrix}
	4 & -1 & 6 \\
	2 & 1 & 6 \\
	2 & -1 & 8
\end{bmatrix}
		$, then $\lambda = 2$ is an eignvalue. Find the corresponding eigenspace.
	\end{example}
	\begin{soln} \
		\begin{itemize}
			\item $\dim A = 1 \implies$ 1 vector
			\item $\dim A = 2 \implies 2$ linearly independent eigenvectors
			\item $\dim A = 3 \implies 3$ linearly independent eigenvectors
		\end{itemize}
		\begin{align*}
			A\bm{v} - \lambda \bm{v} &= \bm{0} \\
			\begin{bmatrix}
				2 & -1 & 6 \\
				2 & -1 & 6 \\
				2 & -1 & 6
			\end{bmatrix} \bm{v} &= \bm{0}
		\end{align*}
		Thus,  $
\begin{bmatrix}
	\frac{1}{2} \\
	1 \\
	0
\end{bmatrix},
\begin{bmatrix}
	-3 \\
	0 \\
	1
\end{bmatrix}
		$.
	\end{soln}
	\begin{theorem}
		The eigenvalues of a triangular matrix are the diagonal entries.
	\end{theorem}
	\begin{proof}
		Eigenspace is $\kernel (A - \lambda I_n)$. Let $\lambda = a_{ii}$, which is an entry in the diagonal. Then, $(A - a_{ii}I_n) = \mb{v} = \bm{0}$. All diagonal entries will be subtracted by $a_{ii}$ and thus the entry at $i$'th column and $i$'th row will be 0. Thus, $\det (A - a_{ii}I) = 0$, as determinant of triangular matrix is product of diagonal.
	\end{proof}
	\begin{theorem}
		If $v_1, \dots, v_p$ are eigenvectors corresponding to distinct eigenvalues $\lambda_1, \dots, \lambda_p$ of an $n \times n$ matrix $A$, then $v_1, \dots, v_p$ are linearly independent.
	\end{theorem}
	\begin{proof}
			
	\end{proof}
\end{document}
