\documentclass[11pt]{scrartcl}
\usepackage[sexy]{../../evan}
\usepackage{graphicx}

\definecolor{dg}{RGB}{2,101,15}
\newtheoremstyle{dotlessP}{}{}{}{}{\color{dg}\bfseries}{}{ }{}
\theoremstyle{dotlessP}
\newtheorem{property}[theorem]{Property}

\newtheoremstyle{dotlessN}{}{}{}{}{\color{teal}\bfseries}{}{ }{}
\theoremstyle{dotlessN}
\newtheorem{notation}[theorem]{Notation}
% Shortcuts
\DeclarePairedDelimiter\ceil{\lceil}{\rceil} % ceil function

\DeclarePairedDelimiter\paren{(}{)} % parenthesis

\newcommand{\df}{\displaystyle\frac} % displaystyle fraction
\newcommand{\qeq}{\overset{?}{=}} % questionable equality

\newcommand{\Mod}[1]{\;\mathrm{mod}\; #1} % modulo operator

\newcommand{\comp}{\circ} % composition

% Text Modifiers
\newcommand{\tbf}{\textbf}
\newcommand{\tit}{\textit}

% Sets
\DeclarePairedDelimiter\set{\{}{\}}
\newcommand{\unite}{\cup}
\newcommand{\inter}{\cap}

\newcommand{\reals}{\mathbb{R}} % real numbers: textbook is Z^+ and 0
\newcommand{\ints}{\mathbb{Z}}
\newcommand{\nats}{\mathbb{N}}
\newcommand{\complex}{\mathbb{C}}
\newcommand{\tots}{\mathbb{Q}}

\newcommand{\degree}{^\circ}

% Counting
\newcommand\perm[2][^n]{\prescript{#1\mkern-2.5mu}{}P_{#2}}
\newcommand\comb[2][^n]{\prescript{#1\mkern-0.5mu}{}C_{#2}}

% Relations
\newcommand{\rel}{\mathcal{R}} % relation

\setlength\parindent{0pt}

% Directed Graphs
\usetikzlibrary{arrows}
\tikzset{vertex/.style = {shape=circle,draw,minimum size=2em}}
\tikzset{svertex/.style = {shape=circle,draw,minimum size=.05em,font=\tiny}}
\tikzset{edge/.style = {->,> = latex'}}
\tikzset{dedge/.style = {-> = latex'}}
\tikzset{dot/.style={inner sep=1.5pt,circle,draw,fill}}

% Contradiction
\newcommand\contradiction{\mathbin{\mathpalette\xhash\relax}}
\newcommand{\xhash}[2]{\ooalign{%
Ê $#1\xxhash{#1}{-45}$\cr
Ê $#1\xxhash{#1}{45}$\cr
Ê }%
}
\newcommand{\xxhash}[2]{\rotatebox[origin=c]{#2}{$#1\parallel$}}

\title{MATH 22A: Vector Calculus and Linear Algebra}
\subtitle{PSet 0}
\author{Denny Cao}
\date{\today}
%++++++++++++++++++++++++++++++++++++++++
% title stuff
\usepackage{titling}
\renewcommand\maketitlehooka{\null\mbox{}\vfill}
\renewcommand\maketitlehookd{\vfill\null}
\makeatletter
\renewcommand{\maketitle}{\bgroup\setlength{\parindent}{0pt}
	\begin{flushleft}
		\large\textbf{\@title} \\ \vskip 0.2cm
		\begingroup
		\fontsize{14pt}{12pt}\selectfont
		\title
		\\
		\normalsize Week 1: September 6, 2023 
		\endgroup \vskip 0.3cm
		\normalsize Pset Due: September 13, 2023 \hfill\rlap{}\textbf{Denny Cao} \\ \vskip 0.1cm
		\hrulefill
	\end{flushleft}\egroup
}
\makeatother

\renewcommand{\theques}{\thesection.\alph{ques}} % Change subtheo counter for alpha output
\declaretheorem[style=basehead,name=Answer,sibling=theorem]{ans}
\renewcommand{\theans}{\thesection.\alph{ans}}
\begin{document}
\maketitle
\pagestyle{plain}
\section{Administrative Stuff}
\begin{itemize}
	\item \textbf{Office Hours}: Sunday -- Tuesday (4-6pm) and Thursday (7-9pm) at Northwest B101
	\item Homework available on Wednesdays and due next Wednesday
	\begin{itemize}
		\item Submit on Gradescope
	\end{itemize} 
	\item Midterms: October 12 and November 9 at 4-5:30pm
	\item Finals: December 12
\end{itemize}
\begin{remark}
	[$\sqrt{2}$ Myth]
Pythagorians discovered that it was not possible to be a rational number in  $\frac{p}{q}$ form, which was against their beliefs of everything being able to be expressed as integers.
\end{remark}
\section{Linear Algebra}
\subsection{Applications of Linear Algebra}
We can solve this simple system with matrices and augmented matrices:
\begin{align*}
	10x + 5y &= 3 \\
	5x - 2y &= 7
\end{align*}
\[
\begin{bmatrix}
	10 & 5 & 3 \\
	5 & -2 & 7 \\
\end{bmatrix}
\]
We can also represent graphs as matrices with adjacency matrix.
\begin{definition}
	[Linear Equation] A \textbf{linear equation} in the variables $x_1,...,x_n$ is an equation in the following form:
	\[
	a_1x_1 + a_2x_2 + ... + a_n x_n = b
	\] 
	where $b$ and the coefficients $a_1, ..., a_n$ are real or complex numbers. A \textbf{system of linear equations} is a collection of one or more linear equations involving the same variables. A \textbf{solution} to the linear system is a list of $(s_1,...,s_n)$ of numbers that are solutions to each of the linear equations in the system.
\end{definition}
Fundamental questions for systems:
\begin{itemize}
	\item Does there exist at least one solution?
	\item If solution exists, how can we find it?
	\item If solution exists, is it unique?
\end{itemize}
\begin{definition}
	[Unique and Inconsistent]
	A linear system is \textbf{unique} if there is a solution and otherwise, \textbf{inconsistent} (if there is no solution).
\end{definition}
\begin{fact}
	A linear system has either no solution, exactly one solution, or infinitely many.	
\end{fact}
\begin{definition}
	[Elementary Row Operations]
	The following elementary row operations preserve solutions:
	\begin{itemize}
		\item Replace one row by the sum of itself and a multiple of another	
		\item Interchange two rows (Swap)
		\item Multiply all entries in a row by a non-zero constant (Scaling a vector)
	\end{itemize}
\end{definition}
\begin{definition}
	[Row Equivalent]
	Two matrices are called \textbf{row equivalent} if there is a sequence of elementary row operations changing one matrix to another.
\end{definition}
\begin{fact}
	If the augmented matrices of two linear systems are row equivalent, then they have the same solution set.
\end{fact}
\end{document}

