\documentclass[11pt]{scrartcl}
\usepackage[sexy]{../../../evan}
\usepackage{float}
\usepackage{graphicx}
\usepackage{bm}
\usepackage{pgfplots}
\usetikzlibrary{calc}
\usetikzlibrary{decorations,calligraphy}
\usetikzlibrary{matrix,decorations.pathreplacing, calc, positioning,fit, bending}
\definecolor{dg}{RGB}{2,101,15}
\newtheoremstyle{dotlessP}{}{}{}{}{\color{dg}\bfseries}{.}{ }{}
\theoremstyle{dotlessP}
\newtheorem{property}[theorem]{Property}
\newtheorem{axiom}{Axiom}
\newtheoremstyle{dotlessN}{}{}{}{}{\color{teal}\bfseries}{}{ }{}
\newtheorem{sol}{Solution}[section]
\newtheoremstyle{dotlessN}{}{}{}{}{\color{teal}\bfseries}{}{ }{}
\theoremstyle{dotlessN}
\newtheorem{notation}[theorem]{Notation}
% Shortcuts
\DeclarePairedDelimiter\ceil{\lceil}{\rceil} % ceil function

\DeclarePairedDelimiter\paren{(}{)} % parenthesis

\newcommand{\df}{\displaystyle\frac} % displaystyle fraction
\newcommand{\qeq}{\overset{?}{=}} % questionable equality

\newcommand{\Mod}[1]{\;\mathrm{mod}\; #1} % modulo operator

\newcommand{\comp}{\circ} % composition

\newcommand{\lra}{\leftrightarrow}

% Text Modifiers
\newcommand{\tbf}{\textbf}
\newcommand{\tit}{\textit}

% Sets
\DeclarePairedDelimiter\set{\{}{\}}
\newcommand{\unite}{\cup}
\newcommand{\inter}{\cap}

\newcommand{\reals}{\mathbb{R}} % real numbers: textbook is Z^+ and 0
\newcommand{\ints}{\mathbb{Z}}
\newcommand{\nats}{\mathbb{N}}
\newcommand{\complex}{\mathbb{C}}
\newcommand{\tots}{\mathbb{Q}}
\newcommand{\smin}{\setminus}
\newcommand{\degree}{^\circ}
\newcommand{\xor}{\oplus}
\newcommand{\powset}{\mathcal{P}}
% Counting
\newcommand\perm[2][^n]{\prescript{#1\mkern-2.5mu}{}P_{#2}}
\newcommand\comb[2][^n]{\prescript{#1\mkern-0.5mu}{}C_{#2}}

% Relations
\newcommand{\rel}{\mathcal{R}} % relation

\setlength\parindent{0pt}

% Directed Graphs
\usetikzlibrary{arrows}
\tikzset{vertex/.style = {shape=circle,draw,minimum size=2em}}
\tikzset{svertex/.style = {shape=circle,draw,minimum size=.05em,font=\tiny}}
\tikzset{edge/.style = {->,> = latex'}}
\tikzset{dedge/.style = {-> = latex'}}
\tikzset{dot/.style={inner sep=1.5pt,circle,draw,fill}}

% Linear Algebra
\newcommand{\nul}{\text{Nul}}
\newcommand{\col}{\text{Col}}
\newcommand{\row}{\text{Row}}
\newcommand{\adj}{\text{adj}}
\newcommand{\spa}[1]{\text{Span}\set*{#1}}
\newcommand{\mb}[1]{\mathbf{#1}}
\newcommand{\poly}{\mathbb{P}}
\newcommand{\basis}{\mathcal{B}}
\newcommand{\tr}{\text{tr}}
\newcommand{\dist}[1]{\text{dist}\paren*{#1}}
\newcommand{\proj}[2]{\text{proj}_{#1}{#2}}
\newcommand{\ip}[2]{\langle #1, #2 \rangle}
% Contradiction
\newcommand{\contradiction}{{\hbox{%
    \setbox0=\hbox{$\mkern-3mu\times\mkern-3mu$}%
    \setbox1=\hbox to0pt{\hss$\times$\hss}%
    \copy0\raisebox{0.5\wd0}{\copy1}\raisebox{-0.5\wd0}{\box1}\box0
}}}
\newcommand{\xxhash}[2]{\rotatebox[origin=c]{#2}{$#1\parallel$}}

\hypersetup{
	linkcolor=magenta
}
\title{MATH 22A: Vector Calculus and Linear Algebra}
\author{Denny Cao}
\date{\today}
%++++++++++++++++++++++++++++++++++++++++
% Heading and Footer
%++++++++++++++++++++++++++++++++++++++++
% title stuff
\makeatletter
\renewcommand*\env@matrix[1][*\c@MaxMatrixCols r]{%
  \hskip -\arraycolsep
  \let\@ifnextchar\new@ifnextchar
  \array{#1}}
\makeatother

\renewcommand{\maketitle}{\bgroup\setlength{\parindent}{0pt}
	\begin{flushleft}
		\large\textbf{MATH 22A: Vector Calculus and Linear Algebra} \\ \vskip 0.2cm
		\begingroup
		\fontsize{14pt}{12pt}\selectfont
		\title
		\\
		Problem Set 12
		\endgroup \vskip 0.3cm
		Due: Tuesday, December 5, 2023 12pm \hfill\rlap{}\textbf{Denny Cao} \\ \vskip 0.1cm
		\hrulefill
	\end{flushleft}\egroup
}

\begin{document}
\maketitle
\pagestyle{plain}
\section*{Collaborators}
\section{Computational Problems}
\begin{sol}
	We list the values of each polynomial as a vector in $\reals^5$:
	\[
	\begin{bmatrix}
		1 \\
		1 \\
		1 \\
		1 \\
		1
	\end{bmatrix},
	\begin{bmatrix}
		-2 \\
		-1 \\
		0 \\
		1 \\
		2
	\end{bmatrix}, 
	\begin{bmatrix}
		4 \\
		1 \\
		0 \\
		1 \\
		4
	\end{bmatrix},
	\begin{bmatrix}
		-8 \\
		-1 \\
		0 \\
		1 \\
		8
	\end{bmatrix},
	\begin{bmatrix}
		16 \\
		1 \\
		0 \\
		1 \\
		16
	\end{bmatrix}
	\] 
	We then use the Gram-Schmidt Process to obtain an orthogonal basis. Let $p_i$ denote the $i$-th polynomial above. Let $v_i$ denote the $i$-th vector in the orthogonal basis for $\poly_4$:
	\begin{align*}
		p_0(t) &= 1 &\quad 
		p_1(t) &= t - \frac{\ip{t}{p_0}}{\ip{p_0}{p_0}}p_0(t) = t - 0p_0(t)\\
		v_0	&= 
		\begin{bmatrix}
			1 \\
			1 \\
			1 \\
			1 \\
			1
		\end{bmatrix} &\quad
		v_1	&= 
			\begin{bmatrix}
				-2 \\
				-1 \\
				0 \\
				1 \\
				2
			\end{bmatrix} \\
		p_2(t) &= t^2 - \frac{\ip{t^2}{p_0}}{\ip{p_0}{p_0}}p_0(t) - \frac{\ip{t^2}{p_1}}{\ip{p_1}{p_1}}p_1(t) &\quad 
		p_3(t) &= t^3 - 
		\frac{\ip{t^3}{p_0}}{\ip{p_0}{p_0}}p_0(t) - \frac{\ip{t^3}{p_1}}{\ip{p_1}{p_1}}p_1(t) - \frac{\ip{t^3}{p_2}}{\ip{p_2}{p_2}}p_2(t)
		\\
			&= 
		t^2 - 2p_0(t) &\quad &= t^3 - \frac{17}{5}p_1(t)\\
			v_2&= 
			\begin{bmatrix}
				2 \\
				-1 \\
				-2 \\
				-1 \\
				2
			\end{bmatrix} &\quad v_3 &= 
			\begin{bmatrix}
				-6/5 \\
				12/5 \\
				0 \\
				17/5 \\
				74/5
			\end{bmatrix} \\
		\end{align*}
		\begin{align*}
			p_4(t) &= t^4 - \frac{\ip{t^4}{p_0}}{\ip{p_0}{p_0}}p_0(t) - \frac{\ip{t^4}{p_1}}{\ip{p_1}{p_1}}p_1(t) - \frac{\ip{t^4}{p_2}}{\ip{p_2}{p_2}}p_2(t) - \frac{\ip{t^4}{p_3}}{\ip{p_3}{p_3}}p_3(t)	\\
				   &= t^4 - \frac{36}{5}p_0(t) - 13p_2(t) \\
				   &= 
				   \begin{bmatrix}
				   	-216/5 \\
					-96/5 \\
					0 \\
					-96/5 \\
					-216/5
				   \end{bmatrix}
		\end{align*}
\end{sol}
\begin{sol}
	\begin{proof}
		To show that $\ip{u}{v} = T(u) \cdot T(v)$ defines an inner product on $V$, we show that, for all $u,v,w \in V$ and all scalars $c$:
		\begin{itemize}
			\item $\ip{u}{v} = \ip{v}{u}$: 
				\begin{align*}
					\ip{u}{v} &= T(u) \cdot T(v) \\
							  &= T(v) \cdot T(u) \\
							  &= \ip{v}{u}
				\end{align*}
			\item $\ip{u+v}{w} = \ip{u}{w} + \ip{v}{w}$: 
				\begin{align*}
					\ip{u + v}{w} &= (T(u) + T(v))\cdot T(w) \\
								  &= T(u)\cdot T(w) + T(v)\cdot T(w) \\
								  &= \ip{u}{w} + \ip{v}{w}
				\end{align*}
			\item $\ip{cu, v} = c\ip{u}{v}$:
				\begin{align*}
					\ip{cu}{v} &= cT(u) \cdot T(v) \\
							   &= c(T(u) \cdot T(v)) \\
							   &= c\ip{u}{v} 
				\end{align*}
		\item $(\ip{u}{u} \geq 0) \land (\ip{u}{u} = 0 \iff u = 0$):
			\begin{align*}
				\ip{u}{u} &= T(u) \cdot T(u) \geq 0 \\
				\ip{u}{u} &= T(u) \cdot T(u) = 0 \iff T(u) = 0 \iff u = 0
			\end{align*}
		\end{itemize}
		As all axioms are satisfied, the proof is complete.
	\end{proof}
\end{sol}
\end{document}
