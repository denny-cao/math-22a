\documentclass[11pt]{scrartcl}
\usepackage[sexy]{../../../evan}
\usepackage{graphicx}

\definecolor{dg}{RGB}{2,101,15}
\newtheoremstyle{dotlessP}{}{}{}{}{\color{dg}\bfseries}{}{ }{}
\theoremstyle{dotlessP}
\newtheorem{property}[theorem]{Property}

\newtheoremstyle{dotlessN}{}{}{}{}{\color{teal}\bfseries}{}{ }{}
\theoremstyle{dotlessN}
\newtheorem{notation}[theorem]{Notation}
% Shortcuts
\DeclarePairedDelimiter\ceil{\lceil}{\rceil} % ceil function

\DeclarePairedDelimiter\paren{(}{)} % parenthesis

\newcommand{\df}{\displaystyle\frac} % displaystyle fraction
\newcommand{\qeq}{\overset{?}{=}} % questionable equality

\newcommand{\Mod}[1]{\;\mathrm{mod}\; #1} % modulo operator

\newcommand{\comp}{\circ} % composition

% Text Modifiers
\newcommand{\tbf}{\textbf}
\newcommand{\tit}{\textit}

% Sets
\DeclarePairedDelimiter\set{\{}{\}}
\newcommand{\unite}{\cup}
\newcommand{\inter}{\cap}

\newcommand{\reals}{\mathbb{R}} % real numbers: textbook is Z^+ and 0
\newcommand{\ints}{\mathbb{Z}}
\newcommand{\nats}{\mathbb{N}}
\newcommand{\complex}{\mathbb{C}}
\newcommand{\tots}{\mathbb{Q}}

\newcommand{\degree}{^\circ}

% Counting
\newcommand\perm[2][^n]{\prescript{#1\mkern-2.5mu}{}P_{#2}}
\newcommand\comb[2][^n]{\prescript{#1\mkern-0.5mu}{}C_{#2}}

% Relations
\newcommand{\rel}{\mathcal{R}} % relation

\setlength\parindent{0pt}

% Directed Graphs
\usetikzlibrary{arrows}
\tikzset{vertex/.style = {shape=circle,draw,minimum size=2em}}
\tikzset{svertex/.style = {shape=circle,draw,minimum size=.05em,font=\tiny}}
\tikzset{edge/.style = {->,> = latex'}}
\tikzset{dedge/.style = {-> = latex'}}
\tikzset{dot/.style={inner sep=1.5pt,circle,draw,fill}}

% Contradiction
\newcommand{\contradiction}{{\hbox{%
    \setbox0=\hbox{$\mkern-3mu\times\mkern-3mu$}%
    \setbox1=\hbox to0pt{\hss$\times$\hss}%
    \copy0\raisebox{0.5\wd0}{\copy1}\raisebox{-0.5\wd0}{\box1}\box0
}}}
\newcommand{\xxhash}[2]{\rotatebox[origin=c]{#2}{$#1\parallel$}}

\title{MATH 22A: Vector Calculus and Linear Algebra}
\subtitle{PSet 1}
\author{Denny Cao}
\date{\today}
%++++++++++++++++++++++++++++++++++++++++
% Heading and Footer
%++++++++++++++++++++++++++++++++++++++++
% title stuff
\makeatletter
\renewcommand{\maketitle}{\bgroup\setlength{\parindent}{0pt}
	\begin{flushleft}
		\large\textbf{\@title} \\ \vskip 0.2cm
		\begingroup
		\fontsize{14pt}{12pt}\selectfont
		\title
		\\
		Problem Set 1
		\endgroup \vskip 0.3cm
		Due: Wednesday, September 13, 2023 12pm \hfill\rlap{}\textbf{Denny Cao} \\ \vskip 0.1cm 
		\hrulefill
	\end{flushleft}\egroup
}
\makeatother

\begin{document}
\maketitle
\pagestyle{plain}
Collaborators: May Ng, Nina Khera, Charlie Chen, Jaray Liu
\section{Computation Questions}
\makeatletter
\renewcommand*\env@matrix[1][*\c@MaxMatrixCols c]{%
  \hskip -\arraycolsep
  \let\@ifnextchar\new@ifnextchar
  \array{#1}}
\makeatother

\begin{ques}
	Do the three planes below have a common point of intersection?
	\begin{align*}
		x_1 + 2x_2 + x_3 &= 4 \\
		x_2 - x_3 &= 1 \\
		x_1 + x_3 &= 0
	\end{align*}
\end{ques}
\textbf{Solution}

	\textbf{Yes; the three planes intersect at $(-1,2,1)$.}
	\begin{align*}
		\begin{bmatrix}[ccc|c]
			1 & 2 & 1 & 4 \\
			0 & 1 & -1 & 1 \\
			1 & 0 & 1 & 0 \\
		\end{bmatrix}
		\intertext{$\sim R_1 - R_3 \to R_1$.}
		\begin{bmatrix}[ccc|c]
			0 & 2 & 0 & 4 \\
			0 & 1 & -1 & 1 \\
			1 & 0 & 1 & 0 \\
		\end{bmatrix}
		\intertext{$\sim R_3 \leftrightarrow R_1, R_2 \leftrightarrow R_3$.}
		\begin{bmatrix}[ccc|c]
			1 & 0 & 1 & 0 \\
			0 & 2 & 0 & 4 \\
			0 & 1 & -1 & 1 \\
		\end{bmatrix}
		\intertext{$\sim R_2/2 \to R_2$.}
		\begin{bmatrix}[ccc|c]
			1 & 0 & 1 & 0 \\
			0 & 1 & 0 & 2 \\
			0 & 1 & -1 & 1 \\
		\end{bmatrix}
		\intertext{$\sim R_2 - R_3 \to R_3$.}
		\begin{bmatrix}[ccc|c]
			1 & 0 & 1 & 0 \\
			0 & 1 & 0 & 2 \\
			0 & 0 & 1 & 1 \\
		\end{bmatrix}
		\intertext{$\sim R_1 - R_3 \to R_1$.}	
		\begin{bmatrix}[ccc|c]
			1 & 0 & 0 & -1 \\
			0 & 1 & 0 & 2 \\
			0 & 0 & 1 & 1 \\
		\end{bmatrix}
\end{align*}
As systems of equations:
\begin{align*}
	x_1 &= -1 \\
	x_2 &= 2 \\
	x_3 &= 1 
\end{align*}
Thus, the point of intersection of the three planes is $(-1,2,1)$.
\begin{ques}
	Solve the system of linear equations below:
	\begin{align*}
		x_1 - 3x_2 &= 5 \\
		-x_1 + x_2 + 5x_3 &= 2 \\
		x_2 + x_3 &= 0
	\end{align*}
\end{ques}
\textbf{Solution}

	\textbf{The solution for the system of linear equations is: $(2,-1,1)$.}
\begin{align*}
	\begin{bmatrix}[ccc|c]
		1 & -3 & 0 & 5 \\
		-1 & 1 & 5 & 2 \\
		0 & 1 & 1 & 0 \\
	\end{bmatrix}
	\intertext{$\sim R_1 + R_2 \to R_2$.}
	\begin{bmatrix}[ccc|c]
		1 & -3 & 0 & 5 \\
		0 & -2 & 5 & 7 \\
		0 & 1 & 1 & 0 \\
	\end{bmatrix}
	\intertext{$\sim 2R_3 + R_2 \to R_3$.}
	\begin{bmatrix}[ccc|c]
		1 & -3 & 0 & 5 \\
		0 & -2 & 5 & 7 \\
		0 & 0 & 7 & 7 \\
	\end{bmatrix}
	\intertext{$\sim R_3/7 \to R_3$.}
	\begin{bmatrix}[ccc|c]
		1 & -3 & 0 & 5 \\
		0 & -2 & 5 & 7 \\
		0 & 0 & 1 & 1 \\
	\end{bmatrix}
	\intertext{$\sim R_2 - 5R_3 \to R_2$.}
	\begin{bmatrix}[ccc|c]
		1 & -3 & 0 & 5 \\
		0 & -2 & 0 & 2 \\
		0 & 0 & 1 & 1 \\
	\end{bmatrix}
	\intertext{$\sim -R_2/2 \to R_2$.}
	\begin{bmatrix}[ccc|c]
		1 & -3 & 0 & 5 \\
		0 & 1 & 0 & -1 \\
		0 & 0 & 1 & 1 \\
	\end{bmatrix}
	\intertext{$\sim R_1 + 3R_2 \to R_1$.}
	\begin{bmatrix}[ccc|c]
		1 & 0 & 0 & 2 \\
		0 & 1 & 0 & -1 \\
		0 & 0 & 1 & 1 \\
	\end{bmatrix}
\end{align*}
As systems of equations:
\begin{align*}
	x_1 &= 2 \\
	x_2 &= -1 \\
	x_3 &= 1 
\end{align*}
Thus, the solution is $(2,-1,1)$.
\begin{ques}
	Determine if the system of equations below is consistent (no need to solve): 
	\begin{align*}
		x_1 - 2x_4 &= -3 \\
		2x_2 + 2x_3 &= 0 \\
		x_3 + 3x_4 &= 1 \\
		-2x_1 + 3x_2 + 2x_3 + x_4 &= 5
	\end{align*}
\end{ques}
\textbf{Solution}

	\textbf{The system of equations is consistent.}

	\begin{proof}
		We show this by examining if the rightmost column of the augmented matrix in echelon form is a pivot column or not. 
	\begin{align*}
		\begin{bmatrix}[cccc|c]
			1 & 0 & 0 & -2 & -3 \\
			0 & 2 & 2 & 0 & 0 \\
			0 & 0 & 1 & 3 & 1 \\
			-2 & 3 & 2 & 1 & 5 \\
		\end{bmatrix}
		\intertext{$\sim 2R_1 + R_4 \to R_4$.}
		\begin{bmatrix}[cccc|c]
			1 & 0 & 0 & -2 & -3 \\
			0 & 2 & 2 & 0 & 0 \\
			0 & 0 & 1 & 3 & 1 \\
			0 & 3 & 2 & -3 & -1 \\
		\end{bmatrix}
		\intertext{$\sim R_2 - 2R_3 \to R_2$.}
		\begin{bmatrix}[cccc|c]
			1 & 0 & 0 & -2 & -3 \\
			0 & 2 & 0 & -6 & -2 \\
			0 & 0 & 1 & 3 & 1 \\
			0 & 3 & 2 & -3 & -1 \\
		\end{bmatrix}
		\intertext{$\sim R_2/2 \to R_2$.}
		\begin{bmatrix}[cccc|c]
			1 & 0 & 0 & -2 & -3 \\
			0 & 1 & 0 & -3 & -1 \\
			0 & 0 & 1 & 3 & 1 \\
			0 & 3 & 2 & -3 & -1 \\
		\end{bmatrix}
		\intertext{$\sim R_4 - 3R_2 \to R_4$.}
		\begin{bmatrix}[cccc|c]
			1 & 0 & 0 & -2 & -3 \\
			0 & 1 & 0 & -3 & -1 \\
			0 & 0 & 1 & 3 & 1 \\
			0 & 0 & 2 & 6 & 2 \\
		\end{bmatrix}
		\intertext{$\sim R_4 - 2R_3 \to R_4$.}
		\begin{bmatrix}[cccc|c]
			1 & 0 & 0 & -2 & -3 \\
			0 & 1 & 0 & -3 & -1 \\
			0 & 0 & 1 & 3 & 1 \\
			0 & 0 & 0 & 0 & 0 \\
		\end{bmatrix}
	\end{align*}
	As the matrix is in echelon form and there the rightmost column of the augmented matrix is not a pivot column, from Theorem 2 (Existence and Uniqueness Theorem) in Section 1.2 of Linear Algebra and Its Applications, the system is consistent.
	\end{proof}
\begin{ques}
	For each of the matrices below, find the value or values of $h$ such that the matrix is the augmented matrix of a consistent linear system of equations.
\[
	\begin{bmatrix}
		1 & h & -3 \\
		-2 & 4 & 6
	\end{bmatrix} \quad 
	\begin{bmatrix}
		2 & -3 & h \\
		-6 & 9 & 5 \\
	\end{bmatrix}
\] 	
\end{ques}
\textbf{Solution}

	\textbf{The first augmented matrix is consistent for all real numbers $h$ and the second matrix is consistent only when $h = -5/3$.}
	\begin{proof}
		With the first matrix:
		\[
			\sim 2R_1 + R_2 \to R_2
			\begin{bmatrix}[cc|c]
				1 & h & -3 \\
				0 & 4 + 2h & 0 
			\end{bmatrix}
		\]
 As the rightmost column of the matrix is not a pivot column for all real numbers $h$, from Theorem 2 (Existence and Uniqueness Theorem) in Section 1.2 of Linear Algebra and Its Applications, the system is consistent for all real numbers $h$.
	\end{proof}
	\begin{proof}
		With the second matrix:
		\[
			\sim 3R_1 + R_2 \to R_2
			\begin{bmatrix}[cc|c]
				2 & -3 & h \\
				0 & 0 & 5 + 3h 
			\end{bmatrix}
		\]
 As the rightmost column of the matrix is only not a pivot column when $h = -5/3$, from Theorem 2 in Section 1.2 of Linear Algebra and Its Applications, the system is consistent only for $h = -5/3$. When $h=-2$, it is consistent with infinitely many solutions.
	\end{proof}
\begin{ques}
	Find the elementary row operations that transforms the first matrix into the second and then find the reverse row operation that transforms the second matrix into the first.
\[
\begin{bmatrix}	
	1 & 3 & -4 \\
	0 & -2 & 6 \\
	0 & -5 & 9 \\ 
\end{bmatrix} \quad
\begin{bmatrix}
	1 & 3 & -4 \\
	0 & 1 & -3 \\
	0 & -5 & 9 \\
\end{bmatrix}
\] 	
\end{ques}
\textbf{Solution}

	\textbf{To transform the first matrix into the second, the second row can be scaled by a factor of $-1/2$.}	
	\\

	\textbf{To transform the second matrix into the first, the second row can be scaled by a factor of $-2$.}
\begin{ques}
Find the elementary row operations that transforms the first matrix into the second and
then find the reverse row operation that transforms the second matrix into the first.
\[
\begin{bmatrix}
	1 & 2 & -5 & 0 \\
	0 & 1 & -3 & -2 \\
	0 & -3 & 9 & 5 \\
\end{bmatrix} \quad
\begin{bmatrix}
	1 & 2 & -5 & 0 \\
	0 & 1 & -3 & -2 \\
	0 & 0 & 0 & -1 \\
\end{bmatrix}
\] 
\end{ques}
\textbf{Solution}

	\textbf{To transform the first matrix into the second, replace the third row with the sum of itself and 3 times the second row.}
	\\

	\textbf{To transform the second matrix into the first, replace the third row with the sum of itself and -3 times the second row.}
\section{Proof Problems}
\begin{ques}
A census taker approaches a man leaning on his gate and asks about his children. He says, ``I have three children and the product of their ages is thirty-six. The sum of their ages is the number on this gate.'' The census taker does some calculation and claims not to have enough information. The man enters his house, but before closing the door tells the census taker, ``I have to see to my eldest child who is in bed with measles.'' The census taker departs, satisfied. What are their ages? Be sure to fully explain your solution.
\end{ques}
\textbf{Solution}

\textbf{Their ages are 9,2,2.}
\begin{proof}
Let $x_1,x_2,x_3$ denote the ages of the children, were $x_1 \geq x_2 \geq x_3$. From the man's first statement, that ``the product of their ages is thirty-six'', we have:
	\[
	x_1x_2x_3 = 36
	\] 
	We list all possible combinations as well as their sum:
	\begin{center}
	\begin{tabular}{c|c|c|c}
		$x_1$ & $x_2$ & $x_3$ & $x_1 + x_2 + x_3$ (Sum) \\		
		\hline
		36 & 1 & 1 & 38 \\
		18 & 2 & 1 & 21 \\
		12 & 3 & 1 & 16 \\
		9 & 4 & 1 & 14 \\
		9 & 2 & 2 & 13 \\
		6 & 6 & 1 & 13 \\
		6 & 3 & 2 & 11 \\
		4 & 3 & 3 & 10
	\end{tabular}
	\end{center}
	The census taker needs more information after being told that the sum of their ages, $x_1 + x_2 + x_3$, is the number on the gate. The only scenario where this occurs is if the sum is 13, as there are 2 possibilities for the ages of the children where the sum is 13. We are now left with ages 9,2,2 and 6,6,1. 
	\\

	The man then says that his \textbf{eldest} son has measles. Thus, the solution must be one where $x_1 > x_2 \geq x_3$. In the case of 6,6,1, there would be no eldest son as $x_1 = x_2$. This leaves the only possibility, with the children being aged 9,2,2. 
\end{proof}
\begin{ques}
	\textbf{The ORDER of quantifiers matters!} The only difference between the following two mathematical statements is the order in which the variables $x$ and $y$ are quantified. And yet, \textit{one of the statements is true and the other is false!} Decide which one is true, prove it, and explain why the other is false.
	\begin{enumerate}[(a)]
		\item For every real number $x$, there exists a real number $y$ such that we have $y>x$.
		\item There exists a real number $y$ such that for every real number $x$ we have $y>x$.
	\end{enumerate}
\end{ques}
\textbf{Solution}

	\textbf{(a) is true and (b) is false.}

	\begin{proof}
		[Proof that (a) is true.]
		 Let $y = x +  1$. $\forall x(y > x)$. Since for any real number $x$, we can find  a $y$ that is greater than $x$, it is the case that $\forall x \exists y \mid y > x$.
	\end{proof}
	\begin{proof}
		[Proof that (b) is false.]
		Let $x = y + 1$. $\forall y(x > y)$. Since for any real number $y$ we can find an  $x$ that is greater than  $y$, it cannot be the case that  $\exists y \forall x (y > x)$. 
	\end{proof}
\begin{ques}
	\textbf{The power of the contrapositive.} For each of the following statements, write out the equivalent contrapositive statement, and then prove it. If the statement is not explicitly of the form ``if...then...'', please translate it into an ``if...then...'' statement. The solution of part (a) is given as an example.
	\begin{enumerate}[(a)]
		\item ``Any integer whose square is odd must be odd.''
		\item Let $n$ be an integer. ``If 8 does not divide $n^2 - 1$ then $n$ is even.''
		\item Let $n$ be a positive integer. ``If $n=3k+2$ for some integer $k$, then $n$ is not a perfect square.''
	\end{enumerate}
\end{ques}
\textbf{Solution}
	\begin{enumerate}[(a)]
		\setcounter{enumi}{1}	
		\item The equivalent contrapositive is: ``If $n$ is not even, then $8$ divides $n^2 -1$'', or  ``If $n$ is odd, then $8$ divides $n^2-1$.''
		\begin{claim*}
			[b]
			$n$ is odd $\to$ $8 \mid n^2 -1$.
		\end{claim*}
		\begin{proof}
			By definition of odd number, $\exists k \in \ints \mid n = 2k+1$. It follows that $n^2-1$ can be expessed as $(2k+1)^2 - 1 = 4k(k+1)$. We will show that the product of $k+1$ and $k$ is even by examining two cases:
			\begin{enumerate}[i)]
				\item $k$ is even. By definition of even number, $\exists c \in \ints \mid k = 2c$. Thus, $4k(k+1) = 8c(2c + 1)$.
				\item $k$ is odd. By definition of odd number, $\exists c \in \ints \mid k = 2c + 1$. Thus,  $k+1 = 2c+2 = 2(c+1)$. It follows that $4k(k+1) = 4(2(c+1))(2c+1) = 8(c+1)(2c+1)$.
			\end{enumerate}
			From Definition 4.4 from the Book of Proof, $a \mid b$ if $b = ac, c \in \ints$. As 8 is a factor for both cases, the statement that 8 divides $n^2 - 1$ is true. As this statement is equivalent to the original statement, the proof is complete.
		\end{proof}
		\item The equivalent contrapositive is: ``If $n$ is a perfect square, then $n \neq 3k+2$ for some integer $k$.
			\begin{claim*}
				[c]
				$n$ is a perfect square $\to$ $n \neq 3k+2, k \in \ints$.
			\end{claim*}
			\begin{proof}
				By contradiction. Assume for purposes of contradiction that, if $n$ is a perfect square, then $n = 3k+2, k \in \ints$. Then, it follows that  $n^2$ will be a perfect square as well. $n^2 = (3k+2)^2 = 9k^2 + 12k + 4 = 9k^2 + 12k + 2 + 2 = 3\paren*{3k^2 + 4k + \displaystyle \frac{2}{3}} + 2$. $\contradiction$
				\\
				
				We reach a contradiction, as  $3k^2 + 4k + \displaystyle \frac{2}{3} $, $k \in \ints$ is not an integer. \\

				Thus, we have shown by contradiction that, if $n$ is a perfect square, then $n \neq 3k+2, k \in \ints$. Since this is equivalent to the original statement, the proof is complete.
			\end{proof}
	\end{enumerate}
\begin{ques}
	\textbf{Direct or contrapositive?} Prove each of the following statements either directly or by contrapositive. Please write very clear and careful solutions, and, in particular, identify explicitly whether your proof is direct or contrapositive. In each of the following, $a,b,$ and  $c$ are integers.
	\begin{enumerate}[(a)]
		\item If $a$ and $b$ are odd then $a^2 + 3b^2 + ab$ is odd.
		\item If $b^3 -1$ is odd, then $b$ is even.
	\end{enumerate} 
\end{ques}
\textbf{Solution}
	\begin{enumerate}[(a)]
		\item 
			\begin{proof}
				[Direct Proof]
				By definition of odd number, $\exists c \in \ints \mid a = 2c + 1$ and $\exists k \in \ints \mid b = 2k + 1$. Thus, $a^2 + 3b^2 + ab$ can be expressed as $(2c+1)^2 + 3(2k+1)^2 + (2c+1)(2k+1) = 12k^2 + 4c^2 + 14k + 6c + 4ck + 5$. 

				We rewrite this as $12k^2 + 4c^2 + 14k + 6c + 4ck + 4 + 1$. 

				By the associative property of addition, we group: $(12k^2 + 4c^2 + 14k + 6c + 4ck + 4) + 1$. We can factor 2: $2(6k^2 + 2c^2 + 7k + 3c + 2ck + 2) + 1$. Let  $d = 6k^2 + 2c^2 + 7k + 3c + 2ck + 2$. Thus, as $a^2 + 3b^2 + ab$ can be expressed by $2d+1$ and $d \in \ints$, by definition of odd number, $a^2 + 3b^2 + ab$ is odd.
			\end{proof}
		\item 
		\begin{proof}
			[Proof by Contrapositive]
			By contraposition, it is equal to prove that, if $b$ is odd, then $b^3 -1 $ is even.

			By definition of odd number, $\exists k \in \ints \mid b = 2k + 1$.  It follows that $b^3 -1$ can be expressed as $(2k+1)^3 - 1 = 8k^3 + 12k^2 + 6k$. We can factor out 2: $2(4k^3 + 6k^2 + 3k)$. Let $c = 4k^3 + 6k^2 + 3k$. As there exists a $c \in \ints$ such that $b^3 -1 = 2c$, by definition of even number, $b^3 - 1$ is even.

			As this statement is equivalent to the original statement, the proof is complete.
		\end{proof}
	\end{enumerate}
\begin{ques}
	Prove the following theorem:
	\begin{theorem*}
		Consider the linear system given by $ax+by = r$ and $cx+dy=s$. If $ad-bc \neq 0$, then the linear system has exactly one solution. If $ad-bc=0$, then there is no solution or there are infinitely many, depending on the values of $r$ and $s$.
	\end{theorem*}
\end{ques}
\textbf{Solution}
\begin{proof}
	\begin{align*}
		&\begin{cases}
			ax + by = r \\
			cx + dy = s 
		\end{cases} \\  
		=&\begin{cases}
			adx + bdy = rd \\
			bcx + bdy = bs
		\end{cases} 
	\end{align*}
	We subtract the second equation from the first:
	\begin{align*}
		adx - bcx + bdy - bdy &= rd - bs \\	
		x(ad-bc) &= rd - bs 
	\end{align*}
		We consider two cases:
		\begin{enumerate}[i)]
			\item $ad-bc \neq 0$. If $ad-bc \neq 0$, then there exists one solution where $x = \displaystyle \frac{dr-bs}{ad-bc}$ and  $y = \displaystyle \frac{r-ax}{b} = \frac{r-a\paren*{\frac{dr-bs}{ad-bc}}}{b}$. Thus, the first statement that if  $ad-bc \neq 0$, then the linear system has exactly one solution is true.
			\item $ad-bc = 0$. If $ad-bc = 0$, then $0 = dr-bs$. There will be no solution if $dr-bs \neq 0$ and infinitely many if $dr-bs = 0$. Thus, the second statement that if $ad-bc = 0$, then there is no solution or there are infinitely many, depending on the values of $r$ and $s$ is true.
		\end{enumerate}
		Thus, we have shown that the theorem is true.
\end{proof}
\section{Science Problem}
\begin{ques}
	Write and submit a sentence or two for other Math 22a students that summarizes what an investigator should be researching \underline{before} applying any mathematical or statistical techniques to a given data set.	
\end{ques}
\textbf{Answer}

An investigator should investigate how a data set was created by researching the provenance of the data---investigators should research the methods used to capture the data, what possible errors could be present, what assumptions were made, and so much more. This is to ensure that, when comparing data, an investigator is comparing data that is consistent with each other and not conflicting which will result in more accurate conclusions and make correlations more apparent. 
\end{document}
